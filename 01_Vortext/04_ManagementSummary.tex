\pdfbookmark{Management Summary}{summary}
\addchap*{Management Summary}
Die vorliegende Bachelorarbeit befasst sich mit Reinforcement Learning, das ein Teilgebiet des Machine Learning ist. 
Reinforcement Learning und die zugehörigen Methoden zur Lösung sequenzieller Entscheidungsprobleme werden erklärt und von den anderen Teilgebieten des Machine Learning abgegrenzt.
Der Fokus der Arbeit liegt auf dem Teilgebiet des Temporal-Difference Learning und den Algorithmen Q-Learning und Sarsa.
Die Arbeit erläutert das Konzept von Temporal-Difference Learning. Basierend darauf werden Q-Learning und Sarsa erklärt und miteinander verglichen.
Zur weiteren Evaluation der Algorithmen werden diese für das Strategiespiel Tic-Tac-Toe in Java implementiert.
Es wird untersucht, ob beide Algorithmen das Spiel durch Spiele gegen sich selbst erlernen können und welche Auswirkungen verschiedene Hyperparameter auf das Training haben.
Die trainierten Agenten werden auf Basis der Rate optimaler Aktionen während des Trainings und der erreichten Spielstärke verglichen.
Die Auswertung zeigt, dass Q-Learning im Durchschnitt schneller konvergiert und eine höhere Spielstärke erreicht als Sarsa.
