% Generelle QoL Anweisungen ------------------------------------------------------------

% Befehle zu Abkuerzungen --------------------------------------------------------------
% Die Anweisung \, erzeugt einen kurzen Abstand und wird bei Abkuerzungen oder zwischen Zahlen und Masseinheiten verwendet
\newcommand{\bs}{$\backslash$\xspace}
\newcommand{\ua}{\mbox{u.\,a.}\xspace}
\newcommand{\oa}{\mbox{o.\,a.}\xspace}
\newcommand{\bspw}{bspw.\xspace}
\newcommand{\bzw}{bzw.\xspace}
\newcommand{\ca}{ca.\xspace}
\newcommand{\dahe}{\mbox{d.\,h.}\xspace}
\newcommand{\etc}{etc.\xspace}
\newcommand{\eur}[1]{\mbox{#1\,\texteuro}\xspace}
\newcommand{\evtl}{evtl.\xspace}
\newcommand{\ggfs}{ggfs.\xspace}
\newcommand{\Ggfs}{Ggfs.\xspace}
\newcommand{\gqq}[1]{\glqq{}#1\grqq{}}
\newcommand{\idR}{i.d.R.\xspace}
\newcommand{\inkl}{inkl.\xspace}
\newcommand{\exkl}{exkl.\xspace}
\newcommand{\insb}{insb.\xspace}
\newcommand{\usw}{usw.\xspace}
\newcommand{\Vgl}{Vgl.\xspace}
\newcommand{\sogn}{sogn.\xspace}
\newcommand{\zB}{\mbox{z.\,B.}\xspace}
\newcommand{\engl}{engl.\xspace}
\newcommand{\dt}{dt.\xspace}

\newcommand{\splay}{Self-play\xspace}
\newcommand{\qlearning}{Q-Learning\xspace}
\newcommand{\bothAlgs}{Q-Learning und Sarsa\xspace}
\newcommand{\sarsa}{Sarsa\xspace}
\newcommand{\expertplay}{Expert Play\xspace}
\newcommand{\qtable}{Q-Tabelle\xspace}
\newcommand{\wtable}{W-Tabelle\xspace}
\newcommand{\qValue}{Q-Value\xspace}
\newcommand{\qValues}{Q-Values\xspace}
\newcommand{\wValue}{W-Value\xspace}
\newcommand{\wValues}{W-Values\xspace}
\newcommand{\satuple}{SA Tupel\xspace}
\newcommand{\ttt}{Tic-Tac-Toe\xspace}
\newcommand{\statevalueFunktion}{State-Value Funktion\xspace}
\newcommand{\actionValueFunktion}{Action-Value Funktion\xspace}
\newcommand{\afterStateVFunction}{Afterstate-Value Funktion\xspace}
\newcommand{\qFunction}{Q-Funktion\xspace}
\newcommand{\afterstateTable}{Afterstate-Tabelle\xspace}




% Custom Anweisungen ------------------------------------------------------------------
\newcommand\includepdfWithChapter[4]{%
\includepdf[pages={#1}, pagecommand={\chapter{#3}}]{#4}
\includepdf[pages={#2}]{#4}
}


