\subsection{\qlearning}
Q-Learning lernt ebenfalls die optimale Policy durch Schätzung der Action-Value Funktion. 
Im Gegensatz zu Sarsa approximiert Q-Learning jedoch direkt die optimale Action-Value Funktion $q_*$ und somit die optimale Policy $\pi_*$.
Dafür nutzt Q-Learning bei der Aktualisierung seines Schätzwertes $Q(S_t,A_t)$ immer die Aktion $a$ des Folgezustands $S_{t+1}$, die den maximalen $Q$-Value hat.   
Dies ist  unabhängig davon welche Aktion $A_{t+1}$ der Agent schlussendlich ausführt.  \cite[S. 177]{watkinsLearningDelayedRewards1989} 
Das Update für Q-Learning kann daher wie in \cref{eq:ql_update} formuliert werden\cite[S. 131]{suttonReinforcementLearningIntroduction2018}.

\begin{equation}
    \label{eq:ql_update}
    \equationentry{TD Update Sarsa}
    Q(S_t,A_t)\leftarrow Q(S_t,A_t)+\alpha[R_{t+1}+\gamma\max_aQ(S_{t+1},a)-Q(S_t,A_t)]
\end{equation}

Die Policy $\pi$, die zur Interaktion mit der Umgebung genutzt wird, ist jedoch dennoch wichtig, da sie entscheidet welche Samples der Agent erhält. 
Da für die Aktualisierung jedoch immer der maximale Q-Value genutzt wird, ist eine Abnahme der Explorationswahrscheinlichkeit bei Nutzung von $\epsilon$-greedy keine Bedingung für die Konvergenz zur optimalen Policy \cite[S. 131f.]{suttonReinforcementLearningIntroduction2018}.
Wenn die Konvergenzbedingungen von \ac{TD} Prediction eingehalten werden, ist Q-Learning garantiert zur optimalen Policy $\pi_*$ zu konvergieren \cite[S. 286]{watkinsQlearning1992}.

Der Ablauf des Q-Learning Algorithmus ist als Pseudocode formuliert in Algorithmus \ref{algo_qlearning}.\footnote{eigene Darstellung in Anlehnung an \cite[S. 131]{suttonReinforcementLearningIntroduction2018}}

{\centering
\begin{minipage}{0.7\textwidth}
\begin{algorithm}[H]
\SetAlgoLined
\DontPrintSemicolon
\SetKwInput{kwAlgorithmParam}{Algorithm parameters}

\kwAlgorithmParam{step size $\alpha \in (0;1]$, small $\epsilon > 0$}
Initialize $Q(s,a)$, for all $s \in S^+, a \in A(s)$, arbitrarily except that $Q(terminal,\cdot)=0$\;
\BlankLine
\ForEach{episode}{
    Initialize $S$\;
    \ForEach{step of episode}{
        Choose $A$ from $S$ using policy derived from $Q$ (e.g., $\epsilon$-greedy)\;
        Take action $A$, observe $R$, $S'$\;
        $Q(S,A) \leftarrow Q(S,A) + \alpha[R+\gamma \max_{a} Q(S',a)-Q(S,A)]$\;
        $S \leftarrow S'$
    }
}
\caption{Q-Learning}
\label{algo_qlearning}
\end{algorithm}
\end{minipage}
\par
}


