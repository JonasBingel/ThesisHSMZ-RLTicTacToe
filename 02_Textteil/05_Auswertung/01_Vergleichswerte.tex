\section{Vergleichsmaßstab der Spielstärke}
\label{sec:vergleichswerte}
Zur Evaluation spielen die trainierten Agenten jeweils 10.000 Spiele als beide Symbole gegen den Minimax-Algorithmus und einen Spieler mit Zufallsstrategie, Random. 
Damit eine Auswertung der Ergebnisse erfolgen kann, ist es zunächst notwendig einen Vergleichsmaßstab zu schaffen. 
Dafür spielen die Evaluationsgegner, Minimax-Algorithmus und Random, selbst gegeneinander.

\cref{tab:resultmatrix_baseline} enthält die Spielergebnisse jeder Kombination von Minimax und Random.
Die dargestellten Ergebnisse sind das arithmetische Mittel von fünf Durchläufen und konsistent zu den Experimenten in \cite{mirnovi.QLearningTicTacToe2020}.
Minimax spielt gegen sich selbst immer zu einem Unentschieden und gewinnt gegen Random in 97\% der Spiele als X und 78\% der Spiele als O.
Minimax kann gegen Random nicht immer gewinnen, da die Wahrscheinlichkeit besteht, dass Random zufällig optimale Aktionen wählt und das Spiel so in einem Unentschieden endet.

Aus den Testspielen kann zudem die durchschnittliche Wahrscheinlichkeit berechnet werden, mit der eine zufällig gewählte Aktion laut Minimax optimal ist. 
Für das Symbol X beträgt die Wahrscheinlichkeit ungefähr 70\% und für das Symbol O knapp 38\%. 
Die Wahrscheinlichkeit ist abhängig vom Symbol, da beispielsweise laut Minimax jede erste Aktion von X optimal ist. Hingegen muss O bereits in seiner ersten Aktion einen Konter für die Aktion von X wählen. Dies erklärt zudem, wieso im Szenario Random gegen Random das Symbol X mehr als die Hälfte der Spiele gewinnt. Basierend auf diesen Spielergebnissen wurde die Spielstärke beider Evaluationsgegner berechnet. 
Wie in \cref{sec:eval_metrik} beschrieben, ist das Intervall der Spielstärke $[-18.000;9.000]$. 
Minimax erreicht eine Spielstärke von $6917,86 \pm 22,29$ und Random von $-6938 \pm34,67$.

\begin{table}
\centering
\caption{Spielergebnismatrix von Minimax und Spieler mit Zufallsstrategie Random}
\label{tab:resultmatrix_baseline}

\begin{tabular}{llrlr}
\toprule
 & \multicolumn{2}{l}{\textbf{X Minimax}} & \multicolumn{2}{l}{\textbf{X Random}} \\ \midrule
\textbf{O Minimax}  & X Minimax:        & 0,00\% $\pm$    0,00\%            & X Random:         & 0,00\% $\pm$ 0,00\%  \\
                    & O Minimax:        & 0,00\% $\pm$    0,00\%            & O Minimax:        & 77,63\% $\pm$ 4,91\%  \\
                    & Unentschieden:    & 100,00\% $\pm$  0,00\%            & Unentschieden:    & 22,37\% $\pm$ 4,91\%  \\ \cmidrule{2-5}
\textbf{O Random}   & X Minimax:        & 0,00\% $\pm$    0,00\%            & X Random:         & 58,38\% $\pm$ 5,41\%  \\
                    & O Random:         & 0,00\% $\pm$    0,00\%            & O Random:         & 29,00\% $\pm$ 1,91\%  \\
                    & Unentschieden:    & 100,00\% $\pm$  0,00\%            & Unentschieden:    & 12,62\% $\pm$ 3,95\%  \\ \bottomrule
\end{tabular}
\end{table}
