\section{Gamefield}
Die Klasse Gamefield implementiert das Tic-Tac-Toe Spielfeld, wie es in \cref{chap:ttt_encoding} beschrieben wurde. 
Gamefield verwaltet pro Symbol ein Bitboard. 
Die Bitboards $B_X$ und $B_O$ wurden umgesetzt mittels der Java BitSet-Klasse, die einen Vektor von Bits verwaltet. 
Zudem implementiert die BitSet-Klasse die klassischen Logikoperationen, sodass die beschriebene Methodik zur Erfassung der legalen Aktionen und Gewinnprüfung umgesetzt werden können. 

In der Anwendung werden möglichen Aktionen und die Binärrepräsentation $B_{s}$ eines Zustands jeweils in einem int zur Basis 10 gespeichert.
Der primitve Datentyp int umfasst 32 Bits und ist somit ausreichend, um die 18 Bit von $B_{s}$ zu verwalten.  \cref{listing:getState} zeigt die Berechnung von $B_{s}$ und somit die Implementierung der \cref{eq:bitboard_bs}. 
Bevor die zu den Bitboards zugehörigen Zahlen addiert werden, erhält $B_X$ mittels LSHIFT den beschriebenen Offset von 9 Bit, d.h. der Spielfeldgröße, damit keine Überschneidung von $B_X$ und $B_O$ auftritt.

\begin{listing}[h]
\caption{getState-Methode zur Berechnung von $B_S$}
\label{listing:getState}
\inputminted{java}{04_Artefakte/03_Listings/getState.java}
\end{listing}