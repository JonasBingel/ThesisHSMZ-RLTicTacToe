\section{Trainingsaufbau der Agenten}
In den folgenden Abschnitten wird der verwendete Trainingsaufbau erklärt. 
Dies umfasst die Arten von \splay sowie die Auswahl der Hyperparameter, die untersucht werden.
Abschließend werden die genutzten Metriken vorgestellt.

\subsection{\splay}
Zum Training spielt ein Agent wiederholt gegen Gegner. 
Die Spielstärke der Gegner sollte vergleichbar mit der des Agenten sein und während des Trainings steigen \cite[S.1 ff.]{epsteinIdealTrainer1994}.
Da die Bereitstellung unterschiedlicher Gegner ansteigender Spielstärke für komplexere Strategiespiele nicht praktikabel ist, gibt es das Konzept des \splay. \cite[S. 565f.]{szitaReinforcementLearningGames2012} \cite[S. 1ff.]{digiovanniSurveySelfPlayReinforcement2021}
Im \splay lernt der  Agent das Spiel, indem er gegen eine Kopie von sich selbst spielt. 
Dadurch wird stetig gegen einen Gegner gleicher Stärke gespielt, ohne dass dieser zusätzlich implementiert werden muss \cite[S. 565f.]{szitaReinforcementLearningGames2012}.
Jedoch hat \splay das Problem, das beide Agenten gleichzeitig lernen und in lokalen Maxima stecken bleiben können. 
Dies hat zur Folge, dass das Lernen unvollständig ist und die Agenten frühzeitig mit dem Lernen aufhören und den Zustandsraum nicht ausreichend erkunden. \cite[S. 16]{boyanModularNeuralNetworks1992}, \cite[S. 565f.]{szitaReinforcementLearningGames2012}, \cite{bowlingConvergenceNoRegretMultiagent2004}

Eine Möglichkeit den Zustandsraum repräsentativer zu proben, ist die Agenten abwechselnd lernen zu lassen. 
Während ein Agent lernt, nutzt der andere seine derzeitige  greedy Policy\footnote{Hyperparameter werden entsprechend gesetzt: $\alpha = \epsilon = 0$}.
Nach einer festen Anzahl von Episoden, die im Folgenden Batch genannt werden, wechseln die Rollen. \cite{slatern.GameAiWhy} 
Dies wird wiederholt, bis die Trainingsepisodenanzahl erreicht wurde, wie im Algorithmus \ref{algo_selfplay} dargestellt.
Da zu Beginn des Trainings der Agent das Spiel nicht kennt, spielt der lernende Agent im ersten Batch gegen einen Spieler mit Zufallsstrategie. 

Im Rahmen der Bachelorarbeit werden das klassische \splay und das beschriebene \splay mit alternierenden Rollen für \ttt implementiert und in der Auswertung verglichen. 
Aufgrund von Vorexperimenten wird festgelegt, dass in beiden Fällen zum Training 150.000 Episoden gespielt werden. Ein Batch beim alternierendem \splay umfasst 100 Episoden.

{\centering

\begin{minipage}{0.65\textwidth}

\begin{algorithm}[H]
\SetAlgoLined
\DontPrintSemicolon
\SetKwInput{Kwinput}{Input}

\Kwinput{$agentX$, $agentO$, $training\_episodes$}
$learningAgent \leftarrow agentX$ \;
set hyperparameters of both agents according to their role\;
\For{$episode \leftarrow 0$ \KwTo $training\_episodes$}{
    play an episode of tic-tac-toe\;
    \If{episode $\bmod$ batch$\_$size = 0}{
        switch learningAgent\;
        set hyperparameters of both agents according to their role\;
}

}
\caption{Pseudocode alternierendes \splay}
\label{algo_selfplay}
\end{algorithm}
\end{minipage}
\par
}

\subsection{Auswahl der Hyperparameter}
\label{sec:hyperparameter}
Die Hyperparameter können entweder konstant sein oder im Verlauf des Trainings abnehmen. 
Zur Komplexitätsreduktion wird für die Verkleinerung von Hyperparametern nur eine Methode genutzt.
Diese Methode zur degressiven Abnahme wird als Step-Based bezeichnet und ist durch folgende Formel für einen beispielhaften Parameter $\eta$ definiert \cite[S. 3]{parkNovelLearningRate2020} :

\begin{equation}\label{eq:Eq3}\equationentry{Step-Based Parameterabnahme}
   \eta(x) = \eta_{start}(\frac{\eta_{end}}{\eta_{start}})^{\frac{x}{x_{max}}}
\end{equation}

Der Parameter $\eta$  hat in der ersten Episode ($x=0$) den Wert $\eta_{start}$ und in der Episode $x_{max}$ den Wert $\eta_{end}$. 
Mit steigender Episode $x$ wird die Anpassung zunehmend kleiner. 
Zur weiteren Komplexitätsreduktion wird nur eine kleine Untermenge möglicher Parameterkombinationen untersucht, die durch Vorexperimente ausgewählt wurden. 

Als Diskontierungsfaktor $\gamma$ wird $1$ verwendet, da \acs{TTT} ein endliches Spiel und die Diskontierung somit nicht notwendig ist. 
Da die Definition des optimalen Verhaltens bereits durch den Reward erfolgt, wird $\gamma$ nicht weiter betrachtet.

Die Lernrate $\alpha$ muss während des Trainings gemindert werden, um die Konvergenz der Algorithmen zu gewährleisten. 
Jedoch wird diese Konvergenzbedingung nur selten in der Praxis eingehalten. 
Zudem ist in nicht-stationären Umgebungen eine konstante (kleine) Lernrate zu bevorzugen. \cite[S. 33]{suttonReinforcementLearningIntroduction2018} 
Da der Gegner vom Agent als Teil der Umgebung betrachtet wird und im klassischen \splay die Agenten gleichzeitig lernen, kann \acs{TTT} als nicht-stationäre Umgebung aufgefasst werden.  
In Vorexperimenten zeigte sich, dass konstante $\alpha$  genutzt werden können, wenn $\alpha < 0.3$ gilt.
Daher werden $\alpha=0,1$, $\alpha=0,2$ sowie ein $\alpha$, das zu Beginn $1$ ist und dann gegen $0.1$ konvergiert untersucht, 

Der Explorationswahrscheinlichkeit $\epsilon$ muss so gewählt werden, dass möglichst alle \ac{SA Tupel} besucht werden.
Insbesondere für \sarsa ist es jedoch notwendig, dass $\epsilon$ keinen konstant hohen Wert hat, damit Exploitation erfolgen kann. 
Durch Vorexperimente wurde ermittelt, dass dies zuverlässig erfolgt, wenn zu Beginn $\epsilon = 1$ gilt und im Rahmen des Trainings abnimmt und nach zwei Drittel der Episoden $\epsilon = 0.1$ gilt.